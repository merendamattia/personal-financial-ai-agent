\section{User Interface and Interaction Design}
\label{sec:interface}

\subsection{Streamlit-Based Web Application}

The user interface is implemented using Streamlit, a Python framework for building data applications. This choice enables rapid development without requiring JavaScript expertise. The framework provides automatic state management and reactive UI updates, handling much of the complexity of building interactive web applications. Built-in support for Markdown, charts, and interactive components allows developers to focus on business logic rather than UI infrastructure. Seamless integration with Python data science libraries like Pandas, NumPy, and Plotly enables rapid visualization of financial data.

\subsection{Application Structure and Management}

The Streamlit application is organized into several logical sections. Session management uses Streamlit's built-in session state to maintain current conversation history, extracted financial profiles, generated portfolios, user settings including language and provider preference, and analysis results with associated charts. The application is structured as a multi-page application with several distinct pages. The Chat Page provides the main conversational interface where users interact with ChatbotAgent. The Portfolio Analysis Page displays generated portfolios with detailed metrics and visualizations including allocation charts and historical performance. The Settings Page allows users to configure LLM provider, language, and financial parameters including Monte Carlo simulation settings. The Profile Management Page enables users to load, edit, and save financial profiles as JSON files.

\subsection{Conversation Flow and Profile Interaction}

When a new user arrives at the application, the system displays a welcome message in the user's preferred language. ChatbotAgent initiates a greeting and explains available capabilities, helping the user understand what the system can do. The system requests selection of the LLM provider if not previously configured, allowing users to choose based on their privacy preferences and available resources.

The agent then guides users through information collection using a conversational approach. Rather than presenting a form, the agent asks questions about demographics such as age and employment status. The agent inquires about income and expenses to understand the user's financial capacity. The agent explores existing assets and investments to understand current portfolio composition. The agent assesses risk tolerance through behavioral questions rather than abstract risk rating scales. The agent understands financial goals and time horizons to set appropriate expectations. The conversational approach is natural and adaptive, with questions varying based on previous answers, making the interaction feel like speaking with a knowledgeable advisor.

Once sufficient information is gathered, the portfolio generation process begins. The user requests a portfolio recommendation either explicitly or implicitly through the conversation. FinancialAdvisorAgent receives the full conversation context. The agent extracts a FinancialProfile using the LLM with structured output to ensure data quality. Then, it builds a RAG query based on profile constraints. Relevant assets are retrieved from the vector database. The agent generates a portfolio with specific allocation percentages. The system validates the portfolio structure to ensure allocations sum to 100 percent and comply with constraints. The portfolio is presented to the user with comprehensive analysis including expected return, volatility, and diversification metrics.

\subsection{Visualization and Display Components}

The interface includes rich visualizations that help users understand portfolio concepts. Portfolio composition is shown through a pie chart displaying allocation percentages for each asset class or security. Historical returns are visualized as line charts showing 10-year performance of recommended assets, allowing users to see how the suggested portfolio components have performed historically. Risk-return scatter plots show an efficient frontier with assets plotted by volatility on the x-axis and expected return on the y-axis, helping users visualize the risk-return trade-off. Monte Carlo simulation results are displayed as distribution graphs showing possible portfolio values at future time horizons, illustrating the range of outcomes under different market scenarios.

\subsection{Settings, Configuration and Profile Management}

The settings page empowers users to customize their experience. For LLM provider selection, users can choose between Ollama for local privacy-preserving inference, Google Gemini for advanced cloud-based reasoning, or OpenAI for highest-quality responses. Users can configure API keys for cloud providers, select specific models within each provider, and test connectivity to verify configuration before proceeding. Financial parameters can be customized including Monte Carlo simulation parameters such as the number of scenarios and projection years, initial investment amounts, monthly contribution levels, risk tolerance overrides, and asset class preferences. This customization allows users to tailor the analysis to their specific circumstances.

Users can export their extracted financial profile as JSON, enabling them to download a structured record of their financial situation for record-keeping or sharing with professional advisors. Users can also import previously saved profiles to resume analysis or continue work from a previous session. Manual editing of extracted profiles is supported, allowing users to correct information or update their circumstances. The ability to manage profiles increases user confidence in data privacy and gives users a sense of control over their information.
