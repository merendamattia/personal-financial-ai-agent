\section{System Architecture}
\label{sec:architecture}

\subsection{High-Level Overview}

The Personal Financial AI Agent is built on a modular, layered architecture that separates concerns into distinct components: the core AI agents, data models, retrieval systems, and the user interface. This architectural approach follows several key design principles. Modularity ensures that components are loosely coupled and can be replaced or extended independently. Multi-provider support through a provider abstraction layer enables vendor independence, allowing users to switch between Ollama, Google Gemini, and OpenAI without modifying application code. The system prioritizes privacy through optional local inference capabilities without requiring external dependencies. An extensible tool framework allows new tools and agents to be added without modifying existing code. Finally, the stateless design of agent interactions enables horizontal scaling for enterprise deployments.

\subsection{Core Components}

The system is composed of several interrelated components that work together to deliver financial advisory services. The agent layer contains specialized AI agents built on top of the datapizza-ai framework.\footnote{\url{https://github.com/datapizza-labs/datapizza-ai}.} BaseAgent serves as an abstract foundation that encapsulates common agent functionality including initialization, configuration loading, system prompt management, and tool registration. All specialized agents inherit from BaseAgent to ensure consistency and reduce code duplication. ChatbotAgent focuses on natural conversation and financial profile extraction, maintaining conversation history and interpreting user intent while guiding users toward providing relevant financial information. FinancialAdvisorAgent specializes in portfolio generation and analysis, utilizing RAG to access historical financial data and providing evidence-based recommendations.

The data model layer ensures type safety and clear data contracts throughout the system. FinancialProfile captures demographic and financial characteristics including age, employment status, income, expenses, debt, savings, investment experience, risk tolerance, and financial goals. Portfolio represents investment recommendations including asset allocations, expected returns, volatility metrics, and diversification ratios. PACMetrics encodes portfolio analysis metrics including PAC (patrimonio atteso) values, performance indicators, and risk assessment. These models are implemented using Pydantic, which enforces type safety at runtime and provides validation.

The retrieval layer implements retrieval-augmented generation specifically for financial data. RAGAssetRetriever maintains a vector database of historical ETF and stock data, supports semantic search for relevant financial assets, integrates historical price data spanning ten years, and provides structured asset information for recommendations. This component is essential for grounding LLM recommendations in factual, historical information rather than allowing the model to generate potentially inaccurate figures.

The tool layer provides specialized computation capabilities that agents can invoke. Currently, the system includes financial analysis tools for computing financial metrics, historical returns, and risk indicators. Portfolio validation tools verify portfolio structure and compliance with constraints. Future extensions might include tax optimization tools, rebalancing advisors, and risk simulation engines.

\subsection{Multi-Provider LLM Support}

A critical design feature is the abstraction of LLM provider complexity through a unified provider layer. This enables users to select their preferred provider based on individual needs and constraints.

Ollama provides full offline inference with no data sent to external servers, making it ideal for privacy-sensitive applications. However, it requires local model download and provides variable performance depending on hardware capabilities. Google Gemini offers cloud-based API access with advanced reasoning capabilities and integration with the Google ecosystem, but requires API key authentication and incurs usage costs. OpenAI provides industry-standard LLM capabilities with the highest quality responses and fine-grained API control, but requires API key and billing setup with associated costs. The client abstraction layer in the codebase provides a unified interface for all providers, allowing seamless switching at configuration time without modifying application logic.

\subsection{Data-Flow}

The typical interaction flow through the system follows a well-defined sequence. A user initiates conversation with ChatbotAgent, which engages the user in dialogue to gather financial information. The user provides profile details and financial goals through natural conversation. FinancialAdvisorAgent then extracts a structured FinancialProfile from the conversation history using the LLM with structured output capabilities. The agent queries RAGAssetRetriever for suitable assets based on the user's profile constraints. The portfolio generation logic combines user preferences with historical data to create a diversified portfolio. The system presents recommendations to the user with comprehensive analysis including historical performance, risk metrics, and diversification ratios. The user can refine preferences and iterate through the analysis cycle, with the agent updating recommendations based on new information.
